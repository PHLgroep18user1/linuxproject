%%replace_mail
\documentclass[12pt,a4paper]{report}
\usepackage[dutch]{babel}
\title{Linuxproject asterisk}
\author{Jens Snyers}
\date{\today}
\usepackage{fancyhdr} %pakket voor voettekst en koptekst
\pagestyle{fancy} %declareren van de pagestyle
\fancyhf{} %disabelen van de huidige voet en koptekst
\fancyhead[L]{Project Linux} %Koptekst
\fancyhead[C]{Asterisk}
\fancyhead[R]{Jens Snyers}
\fancyfoot[L]{Provinciale Hogeschool limburg} %Voettekst
\fancyfoot[C]{2TIN}
\fancyfoot[R]{\thepage}
\usepackage{graphicx} %pakket voor afbeeldingen in de tekst te plaatsen
\usepackage{listings} %pakket voor code in de tekst te plaatsen
\usepackage{appendix} %pakket voor de bijlages
\usepackage{hyperref} %pakket voor url
\usepackage{makeidx}
\makeindex

\begin{document}

\begin{titlepage}
\begin{center}
\includegraphics[width=10cm, height=6cm]{asterisk.jpg} \\
\textsc{\LARGE Asterisk}\\[1.5cm]
\textsc{\Large Project Linux}\\[0.5cm]
\emph{Auteur:} Jens Snyers
\vfill
{\large \today}
\end{center}
\end{titlepage}

\tableofcontents

\chapter{ASTERISK\index{Asterisk} ~\cite{ASTERISK}}
\section{Wat is VoIP}
\paragraph{•}
Alvorens ik kan uitleggen wat Asterisk is geef ik even een korte uitleg over VoIP. VoIP is de afkorting voor Voice over Internet Protocol. Dit betekent in het kort: spreken over (data) IP netwerken. De spraak wordt omgezet in digitale informatie en wordt vervolgens verstuurd over het IP netwerk. Hierdoor betaald u voor dit gesprek helemaal niks, waar u zich ook bevindt. Voorwaarde is dan wel dat u beide over een redelijke breedband internet verbinding beschikt. 

\section{Wat is Asterisk}
\paragraph{•}
Asterisk is een software pakket, gebaseerd op GNU/linux software waarmee een volledig functionerende telefooncentrale mee is op te zetten. Hierdoor is het mogelijk om met een pc (server) een VoIP telefooncentrale op te bouwen. Standaard zijn volgende functies ingebouwd: Voicemail service met directory, conferentie gesprekken, Interactief Voice Response, wachtrijen, caller-ID en het ondersteunt verschillende protocollen zoals SIP, H.323 (zowel client als gateway).

\section{Voordelen}
\paragraph{•}
Het mooie van Asterisk is dat het geen speciale hardware nodig heeft om met voice over IP om te gaan. Voor de connectie met de buitenwereld via het traditionele netwerk kun je eenvoudige ISDN of analoge kaarten in de computer gebruiken. Asterisk ondersteunt een handvol kaarten zoals deze op de markt te verkrijgen zijn. Asterisk ondersteunt natuurlijk de kaarten zoals deze worden aangeboden door Digium™. Zij verkopen verschillende ISDN kaarten zoals BRI, PRI en analoog. Het ondersteunt een groote gedeelte van de TDM protocollen zoals wij die hier in Europa kennen. Verder ondersteunt Asterisk de standaard signalering zoals die gebruikt wordt in het zakelijke telefonie systeem. Hiermee wordt een brug geslagen tussen het traditionele telefonie systeem en het toekomstige VoIP.

\section{Beschikbaarheid}
\paragraph{•}
Asterisk is primair ontwikkeld om te draaien op GNU/Linux voor x/86. Overal ter wereld word dit stuk software probleemloos gecompileerd voor OpenBSD, FreeBSD en Mac OS X Jaguar. Op andere platforms en standaard UNIX-like operating systeem moet dit redelijk eenvoudig toegepast kunnen worden voor degene die daar de tijd en de intelligentie voor heeft. 

\section{Mogelijkheden Asterisk}
\paragraph{•}
Asterisk gebaseerde telefonie oplossingen bieden een grootte en flexibele inzet mogelijkheden. Zowel de standaard telefooncentrale mogelijkheden als de nieuwste technologien op het gebied van voip. Asterisk bied u standaard voicemail, conferentie gesprekken, wachtrijen, wachtmuziek en nog vele meer.  In totaal heeft asterisk meer dan 100 verschillende mogelijkheden.

\section{Installatie Procedure}
\paragraph{•}
Alvorens je Asterisk op een bepaald besturingssysteem kan instaleren Zijn er nog andere taken uit te voeren. Zo werkt Asterisk met bepaalde hardware componenten samen, deze moeten dus eerst naar gelang de grote van het systeem worden gemonteerd en geïnstalleerd. Niet alleen moeten er dus VoIP telefoons worden aangekocht maar moet de server ook voorzien zijn van de juiste hardware componenten. Indien alle hardware gekozen en geïnstalleerd is kan men aan de installatie beginnen, Zoals hier voor gezegd is Asterisk gebaseerd op GNU/linux software dus moet de server dit ook zijn. Nadat Linux geïnstalleerd is kan men de Asterisk Applicatie toevoegen. Hierna dient de applicatie nog geconfigureerd te worden en dan kan met het VoIP systeem in gebruik nemen .

\section{Applicaties}
\subsection{IP PBX}
\paragraph{•}
Asterisk is oorspronkelijk gebouwd als een PBX wat vandaag de dag goed is voor 18 percent van de wereldwijde markt van bedrijf telefonie systemen. Niet te min is er een vereiste kennis van bepaalde dingen zoals Linux, telefonie en basis script programming.

\subsection{Gateway}
\paragraph{•}
Gateways gaan verouderde toestellen (PBX’s, ACD’x,..) omzetten naar de moderne VoIP systemen en services. Doordat Asterisk zowel als verouderde toestellen en de moderne VoIP toestellen ondersteund is dit dus een sterke applicatie om gateways en protocol convertors bij te bouwen.

\subsection{IVR}
\paragraph{•}
Interactive Voice Response wordt gebruikt om geld te besparen in steeds dezelfde terug komende taken. (Belgacom gebruikt dit bijvoorbeeld om uw vraag zo goed mogelijk te beantwoorden, door dit aan de juiste mensen door te geven.)  Asterisk maakt het makkelijk om applicaties te bouwen die de toon of spreek input gaat verwerken van de beller.

\subsection{Conference}
\paragraph{•}
Asterisk heeft een applicatie ontwikkeld ‘MeetMe´´ een conference bridge applicatie die tegelijkertijd kan worden gebruikt door honderden personen met maar enkele lijnen van Dialplanscript. 

\section{Support}
\paragraph{•}
Op de website van Asterisk is er een uitstekende ondersteuning voor gebruikers die Astersisk combineren met hardware en producten van Digium.  Zo is er een Asterisk wiki waar je informatie kan op zoeken. Is er een forum waar je met al je vragen terecht kunt en is er een 24/7 online klantendienst.

\section{Voor- en Nadelen}
\paragraph{•}
Hier nog een kleine opsomming van enkele voor en nadelen
\begin{table} [h]
\begin{tabular}{ | l | }
  \hline
  Voordelen\\
  \hline
  Asterisk is vrij te downloaden\\
  Asterisk kan overweg met zeer veel protocollen\\
  Asterisk kan makkerlijk uitbreiden\\
  Asterisk is geschikt voor grote en kleine netwerken\\
  Asterisk kan op elke pc draaien\\
  \hline
\end{tabular}
\caption[Voordelen Asterisk]{In deze tabel worden enkele voordelen getoond van Asterisk.}
\end{table}

\begin{table} [h]
\begin{tabular}{ | l | }
  \hline
  Nadelen\\
  \hline
  Asterisk wordt niet officieel ondersteund\\
  Asterisk kan moeilijk te configureren zijn zonder kennis linux\\
  Asterisk geen GUI\\
  \hline
\end{tabular}
\caption[Nadelen Asterisk]{In deze tabel worden enkele nadelen getoond van Asterisk.}
\end{table}

\appendix
\chapter{Versiebeheersysteem\index{Versiebeheersysteem}}
\section{Instalatie (Github)\index{Github}}
\paragraph{•}
Ik heb gekozen om met github te werken als versiesysteem, dit omdat ik er al een eerdere ervaring met heb gehad. GitHub is tevens een gratis versiesysteem zolang dat jouw bestanden publiek zijn, de toegang tot jouw bestanden kan je echter zelf bepalen.

\subsection{Aanmaken van account en repository}
\paragraph{•}
Het aanmaken van een account en een repository\index{Repository} gebeurt op de website van github \emph{www.github.com}. Door vanboven op "Signup and Pricing" te klikken ga je naar de registratiepagina, hier kies je dan voor een gratis account. Na het aanmaken en bevestigen van het account kan je dan inloggen. Wanneer je bent ingelogd klik je rechts vanboven op je gebruikersnaam en dan komt er volgend scherm. \emph{Figuur \ref{afbeelding1}}
\begin{figure} [h]
\includegraphics[width=15cm, height=2cm]{Afbeelding1.png}
\caption[Aanmaken repository]{Wanneer u hier op "new repository" klikt gaat u naar een pagina waar u een repository kan aanmaken. Nadat alles is ingevuld kan je de repository aanmaken.}
\label{afbeelding1}
\end{figure}

\subsection{Github installeren}
\paragraph{•}
Het instaleren van github in ubuntu gebeurt via de terminal. Door volgende code in te typen wordt git geinstaleerd.\footnote{Binnen in de kaders staat steeds wat er ingevoerd moet worden in de terminal.}
\begin{lstlisting}[frame=BTrl]
$ sudo apt-get install git
\end{lstlisting}
Wanneer git geinstalleerd is kan je beginnen aan de configuratie van het versiebeheersysteem.

\subsection{Configureren van het git systeem}
\paragraph{•}
\begin{enumerate}
    \item 
We gaan eerst na of er al ssh keys aanwezig zijn in het systeem 
\begin{lstlisting}[frame=BTrl]
$ cd ~/.ssh
\end{lstlisting}
Als het "No such file or directory" terug geeft mag naar stap 3 gaan.
    \item 
Omdat er al een ssh key aanwezig is gaan we van deze oude key een back-up maken.
\begin{lstlisting}[frame=BTrl]
$ ls
$ mkdir key_backup
$ cp id_rsa* key_backup
$ rm id_rsa*
\end{lstlisting}
    \item 
Nadat met in de .ssh folder zit gaat men een nieuwe ssh key aanmaken.
\begin{lstlisting}[frame=BTrl]
$ ssh-keygen -t rsa -C "e-mailadress"
\end{lstlisting}
Op de plaats van \emph{e-mailadress} is het de bedoeling dat je uw eigen e-mail meegeeft.
	\item
De aangemaakt key wordt opgeslagen in de .ssh folder. Als volgende stap gaan we deze key toevoegen aan github door bij configuratie een SSH key toe te voegen. Zoals je in de afbeelding kan zien is mijn key al reeds toegevoegd.\emph{Figuur \ref{afbeelding2}}
\begin{figure} [h]
\includegraphics[width=15cm, height=8cm]{Afbeelding2.png}
\caption[Toevoegen Key]{Door vanboven rechts op configuratie te klikken en dan op SSH keys kan je een key toevoegen.}
\label{afbeelding2}
\end{figure}
\end{enumerate}

\subsection{Testen van de verbinding en gebruikersinfo configureren}
\paragraph{•}
Wanneer je volgend commando in de terminal intypd komt er als alles goed gegaan is een tekst waar je dan "yes" op antwoord. Hierna ben je geauthenticeert door github.
\begin{lstlisting}[frame=BTrl]
$ ssh -T git@github.com
\end{lstlisting}
Als volgende kan je nog gebruikersinfo toevoegen door volgende code in de terminal te typen
\begin{lstlisting}[frame=BTrl]
$ git config --global user.name "Voornaam Achternaam"
$ git config --global user.email "e-mailadress"
\end{lstlisting}
Wanneer je alle bovenstaande stappen hebt gedaan kan je documenten synchronizeren met github.

\section{Comitten\index{Committen} (github)}
\paragraph{•}
Het comitten ofwel het synchronizeren tussen de bestanden die op de computer staan en de bestanden die worden bijgehouden door github gebeurt door een reeks commando's in te typen in de terminal.
\begin{lstlisting}[frame=BTrl]
$ git clone git@github.com:
	PHLgroep18user1/linuxproject.git
$ cd linuxproject
$ git remote add upstream 
	git://github.com/PHLgroep18user1/
	linuxproject.git
$ git add asterisk.tex
$ git commit -m 'versie 1'
$ git push origin master
\end{lstlisting}
Hierbij is ``PHLgroep18user1'' de gebruikersnaam van uw github account en asterisk.tex het document dat zal worden gesynchronizeerd.\emph{Figuur \ref{afbeelding3}}
\begin{figure} [h]
\includegraphics[width=15cm,height=6cm]{Afbeelding3.png}
\caption[Historie github]{Nadat de commit succesvol was kan je in github na gaan wie wat en wanneer heeft gemaakt.}
\label{afbeelding3}
\end{figure}

\renewcommand\thechapter{B} % hierdoor is het bijlage B en niet A
\chapter{Configureren Mail Server\index{Mail Server}}
\section{Inleiding}
\paragraph{•}
Ik maak gebruik van de opensource e-mailclient genaamd Mutt. In de volgende alinea's zal ik uitleggen wat Mutt is en hoe je dit moet instaleren.
\paragraph{•}
Mutt is een opensource-, tekstgebaseerde e-mailclient en is beschikbaar voor Linux en Unix. Het werd oorspronkelijk ontworpen door Michael Elkins en is in 1995 vrijgegeven onder de GNU General Public License.

\section{Installeren en configureren Mutt\index{Mutt}}
\paragraph{•}
De installatie gebeurt via de terminal.
\begin{lstlisting}[frame=BTrl]
$ sudo apt-get install mutt
\end{lstlisting}
Wanneer de installatie voltooid is typ je volgend commando in op de terminal.
\begin{lstlisting}[frame=BTrl]
$ gedit .muttrc
\end{lstlisting}
Indien het bestand al bestaat wordt dit geopend, bestaat dit nog niet dan zal dit worden aangemaakt. In het bestand typ je het volgende.
\begin{lstlisting}[frame=BTrl]
set from = "gebruikersnaam@gmail.com"
set realname = "naam voornaam"
set imap_user = "gebruikersnaam@gmail.com"
set imap_pass = "paswoord"
set smtp_url = "smtp://gebruikersnaam@smtp.gmail.com:587/"
set smtp_pass = "paswoord"
\end{lstlisting}

\section{Testen mutt}
\paragraph{•}
We gaan onze installatie en configuratie testen d.m.v een mail te sturen, dit doe je door het volgend commando in de terminal in te typen.
\begin{lstlisting}[frame=BTrl]
$ .muttrc
$ echo "boodschap" | mutt -s "onderwerp" "e-mail adres"
\end{lstlisting}
Indien alles goed geconfigureerd is zal er een e-mail met de boodschap ``boodschap'', met als onderwerp ``onderwerp'' naar het e-mail adres ``e-mail adres'' gestuurd worden.

\renewcommand\thechapter{C} % hierdoor is het bijlage C en niet A
\chapter{Bash script}
\section{Bash script}
\begin{lstlisting}[frame=BTrl]
#!/bin/bash
#inlezen van de gegevens.
echo "Voer een aanspreking in"
read Aanspr
echo "Voer je voornaam in"
read Vnaam
echo "Voer je achternaam in"
read Anaam
echo "Voer je email in"
read EMAIL

#Spaties uit voor en achternaam doen.
voorachter="$Vnaam $Anaam"
bestandsnaam=`echo "$voorachter" | sed 's/ //g'`

#Kopieren van de .tex file
cp asterisks.tex $bestandsnaam.tex

#Veranderen van auteur
sed -i -E 's/\Jens Snyers/\ '$Vnaam' '$Anaam'/g' 
$bestandsnaam.tex

#Builden van .tex file
pdflatex $bestandsnaam
bibtex $bestandsnaam
makeindex $bestandsnaam
pdflatex $bestandsnaam
pdflatex $bestandsnaam

#Email bericht.
SUBJECT="Asterisk"
EMAILMESSAGE="$Aanspr $Vnaam $Anaam \nIn de bijlage 
kan u het PDF-bestand vinden van mijn project."
echo -e $EMAILMESSAGE | mutt -s $SUBJECT -a 
$bestandsnaam.pdf -- $EMAIL
\end{lstlisting}

\listoffigures

\listoftables

\printindex

\bibliography{asterisks}
\bibliographystyle{plain}
\nocite{LaTeX,LaTeX2,Appendix}

\end{document}
